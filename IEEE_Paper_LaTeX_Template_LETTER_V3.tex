% IEEE Paper Template for US-LETTER Page Size (V1)
% Sample Conference Paper using IEEE LaTeX style file for US-LETTER pagesize.
% Copyright (C) 2006-2008 Causal Productions Pty Ltd.
% Permission is granted to distribute and revise this file provided that
% this header remains intact.
%
% REVISION HISTORY
% 20080211 changed some space characters in the title-author block
%
\documentclass[10pt,conference,letterpaper]{IEEEtran}
\usepackage{times,amsmath}
%
\title{Cohana Demo}
%
%\author{%
% author names are typeset in 11pt, which is the default size in the author block
%{First Author{\small $~^{\#1}$}, Second Author{\small $~^{*2}$}, Third Author{\small $~^{\#3}$} }%
% add some space between author names and affils
%\vspace{1.6mm}\\
%\fontsize{10}{10}\selectfont\itshape
% 20080211 CAUSAL PRODUCTIONS
% separate superscript on following line from affiliation using narrow space
%$^{\#}$\,First-Third Department, First-Third University\\
%Address Including Country Name\\
%\fontsize{9}{9}\selectfont\ttfamily\upshape
%
% 20080211 CAUSAL PRODUCTIONS
% in the following email addresses, separate the superscript from the email address 
% using a narrow space \,
% the reason is that Acrobat Reader has an option to auto-detect urls and email
% addresses, and make them 'hot'.  Without a narrow space, the superscript is included
% in the email address and corrupts it.
% Also, removed ~ from pre-superscript since it does not seem to serve any purpose
%$^{1}$\,first.author@first-third.edu\\
%$^{3}$\,third.author@first-third.edu%
% add some space between email and affil
%\vspace{1.2mm}\\
%\fontsize{10}{10}\selectfont\rmfamily\itshape
% 20080211 CAUSAL PRODUCTIONS
% separated superscript on following line from affiliation using narrow space \,
%$^{*}$\,Second Company\\
%Address Including Country Name\\
%\fontsize{9}{9}\selectfont\ttfamily\upshape
% 20080211 CAUSAL PRODUCTIONS
% removed ~ from pre-superscript since it does not seem to serve any purpose
%$^{2}$\,second.author@second.com
%}
%
\begin{document}
\maketitle
%
\begin{abstract} 
For your paper to be published in the conference proceedings, you must
use this document as both an instruction set and as a template into
which you can type your own text.  If your paper does not conform to
the required format, you will be asked to fix it.
\end{abstract}

% NOTE keywords are not used for conference papers so do not populate them
% \begin{keywords}
% keyword-1, keyword-2, keyword-3
% \end{keywords}
%
\section{Introduction}
%
Trends in data analysis in recent decades.

Introduce cohort analysis. Discuss disadvantages on SQL solution.

Raise the health care example.

\emph{\textbf{Example:} A hospital wants to know the side effects of a new medicine A on patients divided by different ages who are diagnosed with disease B. The monitoring on the effects begins after a patient taking the medicine at least 2 times, and is indicated by abnormal values in daily-conducted lab-test C.}

We have designed Cohana (a cohort query engine) in previous paper. Cohana can handle the above query and meet the requirements of general cohort analysis in many areas.

Further, we develop a web interface on top of Cohana to provide users intuitive and powerful cohort query services. It can...

Contribution
\begin{itemize}
\item	C1
\item   C2
\item   C3
\end{itemize}

\section{System Overview}

\subsection{Query Definition}

Define Birth Event, Cohort Selection and Age.

\subsection{System Architecture}

Depict the back end and the front end. (Architecture figure) (data - cohort engine - web interface)

Describe the data format, and configurations needed.

Engine loads data. Web interface interactively shows options according to the data, and sends request to the engine. Engine executes the query and sends back results to be visualised.

\section{Web User Interface}

Prepare data and load data.

Explain options on the web page, and show how to solve the example problem.

Explain the results (x/y axis and values). with support of third-party library, display different types of charts.

\section{Related Work and Conclusion}

MixPanel. Amplitude. 

%\section*{Acknowledgment}



\bibliographystyle{IEEEtran}

\bibliography{IEEEabrv,IEEEexample}

\end{document}
